\abstract
自然言語処理の分野において,対話応答システムに関する研究は盛んに行われているが,
ユーザに共感し,寄り添うことのできる対話システムの実現には至っていない.
それを実現するための方法として,ユーザの抱いている感情を推定するといったことがあげられる.

そこで本研究では,ユーザの発話に対してその文脈を考慮しながら,
発話内に含まれる単語の感情情報を推定する手法を提案する.
単語を感情に密接に関わる感性語,それ以外の一般単語にわけ,
感性語との共起性をもとに一般単語の感情情報を推定する手法はすでに存在する.
本手法では,ここへ文脈考慮性を導入するために,事前学習モデルの BERT から得られる分散表現を利用した.

BERT は周辺の単語の影響を加味した768次元の単語分散表現を出力することが可能である.
この単語分散表現から 10 種類の感情を扱う感情ベクトルを出力するにあたり,
400次元の中間層を設けた3層ニューラルネットワークを採用し,学習を行った.
学習に用いるテキストデータそのものには感情情報に関するラベルを必要としておらず,
感情表現辞典により取得できる感性語とその周辺の一般単語を自動的に抽出して学習することが可能である.
また,出力可能な語彙は BERT で扱うことのできる語彙と同等であるため,
従来の辞書構築型アプローチに比べて対応可能語彙数は大幅に増加した.

これらの性質を利用することにより,同一単語であってもその単語が出現した文脈に応じて,
取得できる感情情報を変化させることができる.
評価実験により,提案手法による単語感情推定の妥当性・文脈考慮性の向上が確認された.

