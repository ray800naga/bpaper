\chapter{実験に用いたシステム入力文}

評価実験を行うのに先立ち,被験者に生成してもらった
システム入力文の一覧を示す.
生成時に使用するよう指定した単語については
太字で示している.

\section{一般単語を用いたシステム入力文}

\begin{longtable}[C]{|l|}
	\caption{本実験における一般単語を用いたシステム入力文の例}
	\label{table:normal_input_list}
	\\
	\endfirsthead
	\multicolumn{1}{l}{\small\it 前ページからの続き}\\
 	\endhead
	\multicolumn{1}{r}{\small\it 表は次ページに続く}\\
	\endfoot
	\multicolumn{1}{r}{\small\it これで終わり}\\
 	\endlastfoot
		\hline
		\textbf{仕事}は人生を豊かにしてくれる. \\
		\hline
		\textbf{仕事}は面倒だし、疲れるから嫌だ. \\
		\hline
		\textbf{学校}がもうすぐ夏休みに入る. \\
		\hline
		もうすぐ夏休みも終わり\textbf{学校}が始まる.	\\
		\hline
		幸福を\textbf{もたらす}鳥.	\\
		\hline
		不運を\textbf{もたらす}行い.	\\
		\hline
		\textbf{試作}を褒めてもらえた. \\
		\hline
		今回の\textbf{試作}は失敗だった.	\\
		\hline
		羽根のように\textbf{軽い}布団を買った.	\\
		\hline
		\textbf{軽い}嘘が大きな問題に発展した.	\\
		\hline
		おばあちゃんからおいしい\textbf{リンゴ}を貰った.	\\
		\hline
		昨日食べた\textbf{リンゴ}が不味すぎた.	\\
		\hline
		\textbf{夏休み}は実家に帰って、昔の友人と会うことができた.	\\
		\hline
		\textbf{夏休み}なのにバイト三昧で疲れた.	\\
		\hline
		芸人のモノ\textbf{真似}がとても面白かった.	\\
		\hline
		自分の癖を他人に悪く\textbf{真似}されて嫌な気持ちになった.	\\
		\hline
		この説明書は図が多くて、準備の\textbf{手順}がわかりやすい.	\\
		\hline
		この椅子は、組み立ての\textbf{手順}が複雑で時間がかかってしまう.	\\
		\hline
		幸運を\textbf{呼び寄せる}.	\\
		\hline
		日頃の行いが不運を\textbf{呼び寄せる}.	\\
		\hline
		\textbf{果物}は甘くて美味しい.	\\
		\hline
		新鮮でない\textbf{果物}を食べた.	\\
		\hline
		昨日買った\textbf{石鹸}は良い香りがする.	\\
		\hline
		職場の\textbf{石鹸}は傷口に沁みる.	\\
		\hline
		飼っている\textbf{海老}が成長して大きくなった.	\\
		\hline
		お寿司屋さんで\textbf{海老}が売り切れてて食べ損なった.	\\
		\hline
		あのお爺さんは町内\textbf{最年長}だが町内で最も活力に満ちた人でもある.	\\
		\hline
		上司や先輩が次々と辞めてしまい、気付けば部署で\textbf{最年長}になってしまった.	\\
		\hline
		新しい\textbf{エアコン}は、部屋がよく冷えてとても快適である.	\\
		\hline
		リビングの\textbf{エアコン}が故障してしまった.	\\
		\hline
		\textbf{地下道}は人が少なくて近道になるのでとても便利だ.	\\
		\hline
		夜の\textbf{地下道}は暗くて少し怖い.	\\
		\hline
		こちらのチームは試合開始からずっと劣勢であったが、なんとか\textbf{建て直し}た.	\\
		\hline
		勝つためには今の状況を\textbf{建て直す}以外に方法はない.	\\
		\hline
		小さな幸福が\textbf{積み重なる}ことで人生が豊かになる.	\\
		\hline
		負債が\textbf{積み重なり}国の財政が危うい.	\\
		\hline		
\end{longtable}

\newpage
\section{感性語を用いたシステム入力文}

\begin{longtable}[C]{|l|}
	\caption{本実験における感性語を用いたシステム入力文の例}
	\label{table:kansei_input_list}
	\\
	\endfirsthead
	\multicolumn{1}{l}{\small\it 前ページからの続き}\\
 	\endhead
	\multicolumn{1}{r}{\small\it 表は次ページに続く}\\
	\endfoot
	\multicolumn{1}{r}{\small\it これで終わり}\\
 	\endlastfoot
	\hline
	兄は私をいきなり\textbf{驚かし}た。 \\
	\hline
	彼は\textbf{笑み}を浮かべた。 \\
	\hline
	海を見ると\textbf{切ない}気持ちになった。 \\
	\hline
	寒くなってくると、こたつが\textbf{恋しい}。\\
	\hline
	最近は平和を\textbf{好む}人が多い。 \\
	\hline
	自分の希望が通らず非常に\textbf{不満}だった。\\
	\hline
	彼は生意気でいつも一言多いけど\textbf{大嫌い}にはなれない。\\
	\hline
	子供が生まれて\textbf{めでたい}と思った。 \\
	\hline
	調子が良いからと言って、相手を\textbf{侮っ}てはいけない。 \\
	\hline
	愚かな行動に\textbf{立腹}した。 \\
	\hline
	かわいい赤ちゃんは\textbf{安らか}に眠っています。 \\
	\hline
	何年経っても変わらない母校を見ると\textbf{感慨}深いものがある。 \\
	\hline
	思うような結果を出せず\textbf{心苦しい}。 \\
	\hline
	生まれ育った街を離れるのは\textbf{名残惜しい}ものだ。 \\
	\hline
	因縁のライバルを倒すことができ、\textbf{嬉し涙}を流した。\\
	\hline
	弟の合格を\textbf{祝する}。 \\
	\hline
	万引きを働いた子供を\textbf{叱りつける}。 \\
	\hline
	ふと見上げると、空は雲ひとつなく\textbf{晴れ渡っ}ていた。 \\
	\hline
\end{longtable}

\newpage
\section{感性多義語を用いたシステム入力文}

\begin{longtable}[C]{|l|}
	\caption{本実験における感性多義語を用いたシステム入力文の例}
	\label{table:kansei_tagi_input_list}
	\\
	\endfirsthead
	\multicolumn{1}{l}{\small\it 前ページからの続き}\\
 	\endhead
	\multicolumn{1}{r}{\small\it 表は次ページに続く}\\
	\endfoot
	\multicolumn{1}{r}{\small\it これで終わり}\\
 	\endlastfoot
	\hline
	秋頃は過ごしやすい気候で\textbf{気持ち}が良い。 \\
	\hline
	天気が良くていい\textbf{気持ち}になった。 \\
	\hline
	彼女と話すと楽しい\textbf{気持ち}になる。\\
	\hline
	サウナで汗をかいて\textbf{気持ち}がリフレッシュした。\\
	\hline
	今日は晴れていて\textbf{気持ち}がいい。\\
	\hline
	相手の\textbf{気持ち}を考えることは得意だ。\\
	\hline
	目の前で大きな事故を見てしまい、悲しい\textbf{気持ち}になった。\\
	\hline
	食べ過ぎて\textbf{気持ち}が悪い。\\
	\hline
	辛い\textbf{気持ち}にさせる相手と無理に一緒にいたせいでますます不幸になった。\\
	\hline
	彼の話を聞いても\textbf{気持ち}が晴れない。\\
	\hline
	人とぶつかって嫌な\textbf{気持ち}になった。\\
	\hline
	彼の\textbf{気持ち}は全くわからないし、知りたくもない。\\
	\hline
	大会で優勝した瞬間、思わず\textbf{涙}がこぼれてしまった。\\
	\hline
	彼の冗談に笑いすぎて\textbf{涙}が出た。\\
	\hline
	久々の再会に\textbf{涙}を流して喜んだ。\\
	\hline
	感動的な映画を見て\textbf{涙}が出た。\\
	\hline
	\textbf{涙}の数だけ強くなれるよ。\\
	\hline
	優秀賞を手にすることができ、思わず\textbf{涙}をこぼした。\\
	\hline
	紛争の悲惨な実態を知り、\textbf{涙}が止まらない。\\
	\hline
	あいつに負けてしまい、あまりの悔しさに\textbf{涙}が止まらない。\\
	\hline
	突然の訃報に\textbf{涙}が止まらない。\\
	\hline
	友達に自分の悪口を言われて\textbf{涙}が出た。\\
	\hline
	彼は彼女へ怒るあまり、目に\textbf{涙}を浮かべた。\\
	\hline
	悲しいことがあり、\textbf{涙}を流した。\\
	\hline
	記念日に家族に日ごろの\textbf{思い}を伝えた。\\
	\hline
	彼は世界一の選手になりたいという\textbf{思い}を持っている。\\
	\hline
	試合に勝ったので嬉しい\textbf{思い}をすることができた。\\
	\hline
	両親に感謝の\textbf{思い}を伝えた。\\
	\hline
	彼の言葉からも、両親を大切にする\textbf{思い}が伝わってくる。\\
	\hline
	彼は他人の\textbf{思い}を軽視しがちである。\\
	\hline
	一つの\textbf{思い}に執着して周りが見えなくなる。\\
	\hline
	自分1人しか残らなかったので寂しい\textbf{思い}をした。\\
	\hline
	うまくいかないことが多く、つらい\textbf{思い}を抱えている。\\
	\hline
	なかなか周りの理解を得られず、しんどい\textbf{思い}をしているだろう。\\
	\hline
\end{longtable}