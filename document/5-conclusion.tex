\chapter{結論}
本論文では,文章に対してその文脈を考慮しながら,
文章内の単語がもつ感情情報を推定する手法を提案した.
ラベル付けのなされていないテキストデータセットから,
感情と密接にかかわる単語である感性語とその周辺単語を自動的に抽出し,
単語感情推定のためのデータセットを自動的に生成した.
また,周辺単語に付与する感情ベクトルを感性語との距離に応じて弱めた場合について検証を行った.
文章をBERTに入力した際に得られる各単語に対応した分散表現を入力とすることにより,
得られる感情ベクトルを文脈に応じた適切な形へ変化させることを目指した.

生成した感情ベクトルについて,ベースライン手法に比べて
より人間が作成したものと近い出力を得られていることが示唆された.
また,一般単語に対するシステムの出力について,
BERTの単語分散表現を用いることにより
文脈を考慮した感情ベクトルを取得できているという評価を受けた.
また様々な感情についてラベル付けがなされている感性多義語については,
感性語との距離に応じた感情ベクトルの重みづけによって
不適切な感情の出力を抑制できているという評価を受けた.

本手法で得られる文脈考慮性を持った単語感情情報は,
文感情推定にミクロな視点を与えることで,さらなる精度の向上をもたらすことを期待している.
また,対話に含まれるトピックについて,どのような捉え方の下で
やり取りがなされているかを把握することが可能となる等,
対話システムがより人間的なふるまいをするのにも役立つと考える.
人間と共感できるコンピュータの実現に向け,
今後も研究を深め,発展させていきたいと考えている.

以下,本研究における課題点について述べる.
\begin{itemize}
	\item 感性多義語に感情ベクトルを与えることの是非
	\par 感性多義語には多くの感情に対してラベル付けがなされている.
	つまり,感性多義語の周辺単語全てにこれらの様々な感情情報が付加されていることになる.
	このことが分散表現から感情情報を出力できるようにネットワークを学習させる際の
	ノイズになっている可能性が考えられる.
	感性多義語を感性語の対象から外した場合に結果がどのように変化するのかを検証する必要性がある.
	\item BERTを改善した各種モデルを用いた場合の検証
	\par BERTは2018年に発表されたモデルであり,本論文執筆時点でもBERTを改良したモデルが
	発表されている.ハイパーパラメータの調整や
	学習用データ量の増加により性能向上を果たしたRoBERTa\cite{roberta}や
	BERTを軽量化したALBERT\cite{albert}などのモデルでも本手法を適用できるのか
	検証していく必要があると考えている.
	\item 出力を一人ひとりのユーザに最適化する必要性
	\par 本研究では,大規模なテキストデータセットから,
	一般単語と感性語の共起性に着目し,一般単語の感情を推定した.
	しかし,同じ言葉であっても一人ひとり抱く感情は異なってくる.
	そこで,本手法に人間によるフィードバックを反映する仕組みを導入することにより,
	ユーザ一人ひとりに最適化された感情推定システムを実現可能であると考える.
	これにより,ユーザに共感しながら悩み相談を行うようなシステム等,
	人間に寄り添った対話システムを構築することができるのではないかと考える.
\end{itemize}