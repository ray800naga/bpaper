\chapter{結論}
本論文では,文章に対してその文脈を考慮しながら,
文章内の単語がもつ感情情報を推定する手法を提案した.
ラベル付けのなされていないテキストデータセットから,
感情と密接にかかわる単語である感性語とその周辺単語を自動的に抽出し,
単語感情推定のためのデータセットを自動的に生成した.
また,周辺単語に付与する感情ベクトルを感性語との距離に応じて弱めた場合について検証を行った.
文章をBERTに入力した際に得られる各単語に対応した分散表現を入力とすることにより,
得られる感情ベクトルを文脈に応じた適切な形へ変化させることを目指した.

生成した感情ベクトルについて,ベースライン手法に比べて
より人間が作成したものと近い出力を得られていることが示唆された.
また,一般単語に対するシステムの出力について,
BERTの単語分散表現を用いることにより
文脈を考慮した感情ベクトルを取得できているという評価を受けた.
また様々な感情についてラベル付けがなされている感性多義語については,
感性語との距離に応じた感情ベクトルの重みづけによって
不適切な感情の出力を抑制できているという評価を受けた.
