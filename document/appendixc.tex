\chapter{ニューラルネットワークの学習について}

\section{学習の様子}
	評価実験で用いたBERT+weight, BERT, embeddingの3手法について,
	それぞれのネットワーク学習時のふるまいを示す.
	以下の図\ref{fig:BERT_plus_weight_loss}~\ref{fig:embedding_loss}は,
	各モデルを学習しているときの
	train loss, validation lossの推移を示したグラフである.

	\begin{figure}[H]
		\centering
		\includegraphics[keepaspectratio, scale=0.8]{./figure/BERT+weight.png}
		\caption{BERT+weightの学習中の各種lossの推移}
		\label{fig:BERT_plus_weight_loss}
	\end{figure}

	\begin{figure}[H]
		\centering
		\includegraphics[keepaspectratio, scale=0.8]{./figure/BERT.png}
		\caption{BERTの学習中の各種lossの推移}
		\label{fig:BERT_loss}
	\end{figure}

	\begin{figure}[H]
		\centering
		\includegraphics[keepaspectratio, scale=0.8]{./figure/embeddings.png}
		\caption{embeddingの学習中の各種lossの推移}
		\label{fig:embedding_loss}
	\end{figure}

	なお,学習についてはEarly Stoppingを採用している.
	5回以上validation lossが更新されなかった場合に学習を停止し,
	それ以前でvalidation lossが最良であったネットワークを用いた.
	この時,各ネットワークの学習に要したエポック数と
	その時のtest lossは以下の表\ref{table:epoch_num}の通りである.

	\begin{table}[H]
		\centering
		\caption{各手法におけるネットワーク学習に要したエポック数とtest loss}
		\label{table:epoch_num}
		\begin{tabular}{|c|c|c|}
			\hline
			& エポック数 & test loss \\
			\hline
			BERT+weight & 23 & 0.0384 \\
			\hline
			BERT & 28 & 0.0792 \\
			\hline
			embedding & 15 & 0.1082 \\
			\hline
		\end{tabular}
	\end{table}

\section{ウィンドウサイズによる性能の変化}
	データセット生成時に感性語の周辺単語を収集する範囲である
	ウィンドウサイズを変化させたときの
	性能の変化について,BERT+weightとBERTのそれぞれで検証した.
	以下の表\ref{table:window_size}にその結果を示す.

	\begin{table}[H]
		\centering
		\caption{ウィンドウサイズの変化によるtest lossの変化}
		\label{table:window_size}
		\begin{tabular}{|c|c|c|c|}
			\hline
			& $W=2$ & $W=3$ & $W=4$ \\
			\hline
			BERT+weight & 0.0294 & 0.0384 & 0.0446 \\
			\hline
			BERT & 0.0670 & 0.0792 & 0.0863 \\
			\hline
		\end{tabular}
	\end{table}

	ウィンドウサイズが大きくなるほど学習完了時のlossが大きくなる傾向にある.
	これは,感情と直接的に関連しているわけではない一般単語の割合が高まり,
	感情推定の難易度が上がっていることが原因として考えられる.

	また,BERT+weightとBERTを比較すると,BERT+weightの方がlossの値が小さい傾向
	にある.
	これは,感性語ではない一般単語に対して強度を弱めた感情ベクトルを与えているためであり,
	システムが出力する値との誤差が小さくなることが原因であると考えられる.
	よって,test lossが小さいこととモデルの性能が高いことは
	必ずしも対応するとは限らないといえる.