\chapter{はじめに}
自然言語処理の分野において,対話応答システムの研究は盛んにおこなわれている.
身の回りにあるスマートフォンや各種家電といった,様々な製品に搭載されるなど実用化も進み,
人々の生活へ急速に浸透してきている.
しかし,その応答がユーザに寄り添っているようなシステムの実現については,
研究されているもののいまだ課題が残っている.
ユーザに寄り添った対話応答システムが実現することにより,
雑談や気遣いのある応答を行うことが可能になると考えられる.
人々の生活に対話応答システムがより一層浸透し,
人間とコンピュータの連携がより深く円滑に行われることで,
情報化社会の進行に伴う諸問題の解決がより一層促されることが予想される.

ユーザに寄り添った対話応答システムに関する研究の方向性の一つとして,
ユーザの感情推定を行うというものがあげられる.
高村らの研究\cite{spin_kyokusei}では,単語の感情極性,
すなわち単語がポジティブとネガティブのどちらのニュアンスを持つのかを,
電子のスピンの方向に見立てることで推定する手法を提案している.
しかし,人間が抱く複雑な感情を表現するのに2成分では情報量が少ないといえる.
武内らの研究\cite{takeuchi}では,多数の単語に対する感情表現辞書を作成した.
単語が持つ複数の感情情報を取得しているが,
一つの単語に対して一対一で感情情報が保存されているため,
単語がどのように使われているのかという点に対する考慮がなされていない.
Kajiwaraらの研究\cite{kajiwara-wrime}では,SNSへ投稿されたテキストに対し感情情報のラベルを付与し,
投稿者の性格情報と組み合わせることにより文章に対する感情情報を推定している.
しかし,この研究で生成されたデータセットは43200件ほどであり,
より高精度な文感情推定を行う上で更なる拡充を行うには,多大なコストが発生すると考えられる.

そこで本研究では,ラベルのないテキストデータからでもデータセットが生成可能な,
文脈考慮性をもった単語感情推定手法を提案する.
データセットの生成は,感情表現辞典\cite{kanjou_hyogen_jiten}を用いて
感情と密接にかかわる単語(以下,感性語とよぶ)とその単語の持つ感情情報を取得し,
感性語周辺の一般単語を自動的に収集することによりなされる.
そのため,テキストデータに感情情報に関するラベル付けがなされていなくても,
大量の単語感情推定用データセットを生成可能である.
また,事前学習モデルのBERT\cite{BERT}をテキストデータの特徴量抽出機として利用した.
BERTは周囲に存在する単語を考慮した文脈考慮性のある単語分散表現を出力することができる.
例えば,"bank"という単語は同じスペルで"銀行"と"土手"の異なる意味を持つ.
BERTでは,同じ単語だが用いられている意味が異なる場合に
分散表現を変化させることができる,という性質を利用した.


評価実験では,システムへ入力する文章を被験者に作成してもらったうえで,
単語感情推定の妥当性,同一単語を含む文章における感情推定の文脈考慮性を評価した.

以下本論文では,第2章で本研究に関連する研究と知見について述べる.
第3章では提案する文脈を考慮した一般単語感情推定手法について,
第4章では評価実験の内容及び結果について述べる.最後に第5章で結論を述べる.
